% Encoding: UTF-8
%%%%%%%%%%%%%%%%%%%%%%%%%%%%%%%%%%%%%%%%%%%%%%%%%%%%%%%%%%%%%%%%%%%%%%%%%%%%%%%

\documentclass[xcolor=pdftex,dvipsnames,table]{beamer}
\setbeamertemplate{navigation symbols}{}%remove navigation symbols
%%%%%%%%%%%%%%%%%%%%%%%%%%%%%%%%%%%%%%%%%%%%%%%%%%%%%%%%%%%%%%%%%%%%%%%%%%%%%%%

% Language settings
% IMPORTANT: If you change the language settings, make sure to run
% cleantex prior to any latex/pdflatex to remove all .aux files, else
% latex might fail!
\usepackage[english]{babel}
%\usepackage[german]{babel}

%\usepackage{graphicx}
\usepackage[utf8]{inputenc}
%\usepackage{mathptmx}

\usepackage{setspace}
\usepackage{color}
\usepackage{listings}
	
	
%%%%%%%%%%%%%%%%%%%%%%%%%%%%%%%%%%%%%%%%%%%%%%%%%%%%%%%%%%%%%%%%%%%%%%%%%%%%%%%

% IMPORTANT/TODO: currently only PDF graphic files should be used, PNG
% or JPEG may harm the Adobe Reader colour palette and result in
% strange display colours.
%\DeclareGraphicsExtensions{.pdf}


%%%%%%%%%%%%%%%%%%%%%%%%%%%%%%%%%%%%%%%%%%%%%%%%%%%%%%%%%%%%%%%%%%%%%%%%%%%%%%%
%% Presentation main settings
%%%%%%%%%%%%%%%%%%%%%%%%%%%%%%%%%%%%%%%%%%%%%%%%%%%%%%%%%%%%%%%%%%%%%%%%%%%%%%%

\title{Systemnahe Informatik}
\subtitle{Übungsgruppe Xeon Phi}
\author{Dominik Walter}
\date{Sommersemester 2018}


%%%%%%%%%%%%%%%%%%%%%%%%%%%%%%%%%%%%%%%%%%%%%%%%%%%%%%%%%%%%%%%%%%

\begin{document}

\section*{Titlepage}
\begin{frame}
  \frametitle{\ }
  \titlepage
\end{frame}

%%%%%%%%%%%%%%%%%%%%%%%%%%%%%%%%%%%%%%%%%%%%%%%%%%%%%%%%%%%%%%%%%%%%%%%%%%%%%%%

\begin{frame}
	\frametitle{Memory Alignment}
	Memory:
	\begin{tabular}{|c|c|c|c|c|c|c|c|c|c|c|c|c|c|c|c|}
		\hline
		0 & 1 & 2 & 3 & 4 & 5 & 6 & 7 & 8 & 9 & 10 & 11 & 12 & 13 & 14 & 15 \\
		\hline
		\hline
		\multicolumn{2}{|c|}{int16} &
		\multicolumn{2}{c|}{int16} &
		\multicolumn{2}{c|}{int16} &
		\multicolumn{2}{c|}{int16} &
		\multicolumn{2}{c|}{int16} &
		\multicolumn{2}{c|}{int16} &
		\multicolumn{2}{c|}{int16} &
		\multicolumn{2}{c|}{int16}\\
		\hline
		\multicolumn{4}{|c|}{int32} &
		\multicolumn{4}{c|}{int32} &
		\multicolumn{4}{c|}{int32} &
		\multicolumn{4}{c|}{int32}\\
		\hline
		\multicolumn{8}{|c|}{int64} &
		\multicolumn{8}{c|}{int64}\\
		\hline
	\end{tabular}
\end{frame}


\begin{frame}[fragile]
	\frametitle{Memory Alignment in C}
	\begin{tabular}{|c|c|}
		\hline
	\begin{lstlisting}	
struct A {
	char c1;
	int64_t i1;
	char c2;
	int32_t i2;
};
	\end{lstlisting} &
	\begin{lstlisting}	
struct B {
	int64_t i1;
	int32_t i2;
	char c1;
	char c2;
};
	\end{lstlisting}
	
	\\
	
	\hline
	\begin{lstlisting}	
struct A {
	char c1;
	char pad[7];
	int64_t i1;
	char c2;
	char pad[3];
	int32_t i2;
};
	\end{lstlisting} &
	\begin{lstlisting}	
struct B {
	int64_t i1;
	int32_t i2;
	char c1;
	char c2;
	char pad[2];
};
	\end{lstlisting} \\
	\hline
	
	\end{tabular}
	\end{frame}

\begin{frame}[fragile]
	\frametitle{Memory Alignment in RISCV}
	\begin{lstlisting}
	
.data 0x100
char_1:		.space 1	#Adresse: 0x100
int64_t_1:	.space 8	#Adresse: 0x101
char2_2:	.space 1	#Adresse: 0x109
int32_t_2:	.space 4	#Adresse: 0x10A

	\end{lstlisting}
\end{frame}

\begin{frame}[fragile]
	\frametitle{Memory Alignment in RISCV}
	\begin{lstlisting}
	
.data 0x100
char_1:		.space 1	#Adresse: 0x100
.align 3				
int64_t_1:	.space 8	#Adresse: 0x108
char_2:		.space 1	#Adresse: 0x110
.align 2
int32_t_2:	.space 4	#Adresse: 0x114
	
	\end{lstlisting}
\end{frame}

\end{document}

%%%%%%%%%%%%%%%%%%%%%%%%%%%%%%%%%%%%%%%%%%%%%%%%%%%%%%%%%%%%%%%%%%%%%%%%%%%%%%%
