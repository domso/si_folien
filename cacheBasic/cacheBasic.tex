% Encoding: UTF-8
%%%%%%%%%%%%%%%%%%%%%%%%%%%%%%%%%%%%%%%%%%%%%%%%%%%%%%%%%%%%%%%%%%%%%%%%%%%%%%%

\documentclass[xcolor=pdftex,dvipsnames,table]{beamer}
\setbeamertemplate{navigation symbols}{}%remove navigation symbols
%%%%%%%%%%%%%%%%%%%%%%%%%%%%%%%%%%%%%%%%%%%%%%%%%%%%%%%%%%%%%%%%%%%%%%%%%%%%%%%

% Language settings
% IMPORTANT: If you change the language settings, make sure to run
% cleantex prior to any latex/pdflatex to remove all .aux files, else
% latex might fail!
\usepackage[english]{babel}
%\usepackage[german]{babel}

%\usepackage{graphicx}
\usepackage[utf8]{inputenc}
%\usepackage{mathptmx}

\usepackage{setspace}
\usepackage{color}
\usepackage{listings}
\usepackage{multirow}
%%%%%%%%%%%%%%%%%%%%%%%%%%%%%%%%%%%%%%%%%%%%%%%%%%%%%%%%%%%%%%%%%%%%%%%%%%%%%%%

% IMPORTANT/TODO: currently only PDF graphic files should be used, PNG
% or JPEG may harm the Adobe Reader colour palette and result in
% strange display colours.
%\DeclareGraphicsExtensions{.pdf}
%\outputdirectory{{./pdfs/}}
\graphicspath{{./img/}}

%%%%%%%%%%%%%%%%%%%%%%%%%%%%%%%%%%%%%%%%%%%%%%%%%%%%%%%%%%%%%%%%%%%%%%%%%%%%%%%
%% Presentation main settings
%%%%%%%%%%%%%%%%%%%%%%%%%%%%%%%%%%%%%%%%%%%%%%%%%%%%%%%%%%%%%%%%%%%%%%%%%%%%%%%

\title{Systemnahe Informatik}
\subtitle{Übungsgruppe Xeon Phi}
\author{Dominik Walter}
\date{Sommersemester 2018}


%%%%%%%%%%%%%%%%%%%%%%%%%%%%%%%%%%%%%%%%%%%%%%%%%%%%%%%%%%%%%%%%%%

\begin{document}

\section*{Titlepage}
\begin{frame}
  \frametitle{\ }
  \titlepage
\end{frame}

%%%%%%%%%%%%%%%%%%%%%%%%%%%%%%%%%%%%%%%%%%%%%%%%%%%%%%%%%%%%%%%%%%%%%%%%%%%%%%%

\begin{frame}
	\frametitle{Cache-Line / Cache-Block}
	 Block-Offset bei $byteweiser\ Adressierung$ (32 Bit):
	\begin{tabular}{|c|c|c|c|c|c|c|c|}
		\hline
		0-3 & 4-7 & 8-11 & 12-15 & 16-19 & 20-23 & 24-27 & 28-31 \\
		\hline
		\hline
		\multicolumn{2}{|c|}{8 Byte} &
		\multicolumn{2}{c|}{8 Byte} &
		\multicolumn{2}{c|}{8 Byte} &
		\multicolumn{2}{c|}{8 Byte} \\
		\multicolumn{2}{|c|}{$\Rightarrow$ 3 Bit} &
		\multicolumn{2}{c|}{$\Rightarrow$ 3 Bit} &
		\multicolumn{2}{c|}{$\Rightarrow$ 3 Bit} &
		\multicolumn{2}{c|}{$\Rightarrow$ 3 Bit} \\
		\hline
		\multicolumn{4}{|c|}{16 Byte} &				\multicolumn{4}{c|}{16 Byte} \\
		\multicolumn{4}{|c|}{$\Rightarrow$ 4 Bit} &				\multicolumn{4}{c|}{$\Rightarrow$ 4 Bit} \\
		\hline
		\multicolumn{8}{|c|}{32 Byte} \\
		\multicolumn{8}{|c|}{$\Rightarrow$ 5 Bit} \\
		\hline
	\end{tabular}
\end{frame}

\begin{frame}
	\frametitle{Cache-Line / Cache-Block}
	Block-Offset bei $wortweiser\ Adressierung$ (32 Bit):
	\begin{tabular}{|c|c|c|c|c|c|c|c|}
		\hline
		0-3 & 4-7 & 8-11 & 12-15 & 16-19 & 20-23 & 24-27 & 28-31 \\
		\hline
		\hline
		\multicolumn{2}{|c|}{2 Wörter} &
		\multicolumn{2}{c|}{2 Wörter} &
		\multicolumn{2}{c|}{2 Wörter} &
		\multicolumn{2}{c|}{2 Wörter} \\
		\multicolumn{2}{|c|}{$\Rightarrow$ 1 Bit} &
		\multicolumn{2}{c|}{$\Rightarrow$ 1 Bit} &
		\multicolumn{2}{c|}{$\Rightarrow$ 1 Bit} &
		\multicolumn{2}{c|}{$\Rightarrow$ 1 Bit} \\
		\hline
		\multicolumn{4}{|c|}{4 Wörter} &				\multicolumn{4}{c|}{4 Wörter} \\
		\multicolumn{4}{|c|}{$\Rightarrow$ 2 Bit} &				\multicolumn{4}{c|}{$\Rightarrow$ 2 Bit} \\
		\hline
		\multicolumn{8}{|c|}{8 Wörter} \\
		\multicolumn{8}{|c|}{$\Rightarrow$ 3 Bit} \\
		\hline
	\end{tabular}
\end{frame}

\begin{frame}
	\frametitle{Cache-Line / Cache-Block}
	Beispiel: 4-Bit Byte-Offset ($\Rightarrow$ 16 Byte pro Cache-Line)
	\begin{tabular}{|c|c|c|c|c|c|c|c|}
		\hline
		1 & F & 2 & 9 & 6 & F & F & C \\
		0001 & 1111 & 0010 & 1001 & 0110 & 1111 & 1111 & 1100\\
		\hline
		\multicolumn{7}{|c|}{Rest} &
		Offset \\
		\hline
	\end{tabular}
\end{frame}

\begin{frame}
	\frametitle{Cache-Satz}
		
	Beispiel: 128 Byte großer Cache mit 16 Byte pro Cache-Line
	\begin{tabular}{|c|c|c|c|}
		\hline
		Slot & Direkt-Abgebildet & 4-fach Satz-Assoziativ & Voll-Assoziativ \\
		 & $\Rightarrow$ 3 Bit & $\Rightarrow$ 1 Bit & $\Rightarrow$ 0 Bit \\
		\hline
		\hline
		0 & 0 & \multirow{4}{*}{0} & \multirow{8}{*}{0} \\
		\cline{2-2}
		1 & 1 &  &  \\
		\cline{2-2}
		2 & 2 &  &  \\
		\cline{2-2}
		3 & 3 &  &  \\
		\cline{2-2}
		\cline{3-3}
		4 & 4 & \multirow{4}{*}{1} &  \\
		\cline{2-2}
		5 & 5 &  &  \\
		\cline{2-2}
		6 & 6 &  &  \\
		\cline{2-2}
		7 & 7 &  &  \\

		\hline
	\end{tabular}
\end{frame}

\begin{frame}
	\frametitle{Cache-Satz}
	Beispiel: Direkt-Abgebildeter Cache
	\begin{tabular}{|c|c|c|c|c|c|cc|c|}
		\hline
		1 & F & 2 & 9 & 6 & F & \multicolumn{2}{c|}{F} & C \\
		0001 & 1111 & 0010 & 1001 & 0110 & 1111 & 1 & 111 & 1100\\
		\hline
		\multicolumn{7}{|c|}{Tag} &
		 Index &
		
		Offset \\
		\multicolumn{7}{|c|}{1F296F, 1} &
		7 &
		
		12 \\
		\hline
	\end{tabular}


	\ \\
	Beispiel: 4-Fach Satz-Assoziativer Cache
	\begin{tabular}{|c|c|c|c|c|c|cc|c|}
		\hline
		1 & F & 2 & 9 & 6 & F & \multicolumn{2}{c|}{F} & C \\
		0001 & 1111 & 0010 & 1001 & 0110 & 1111 & 111 & 1 & 1100\\
		\hline
		\multicolumn{7}{|c|}{Tag} &
		Index &
		
		Offset \\
		\multicolumn{7}{|c|}{1F296F, 111} &
		1 &
		
		12 \\
		\hline
	\end{tabular}
	
	\ \\	
	Beispiel: Voll-Assoziativer Cache
	\begin{tabular}{|c|c|c|c|c|c|c|c|}
	\hline
	1 & F & 2 & 9 & 6 & F & F & C \\
	0001 & 1111 & 0010 & 1001 & 0110 & 1111 & 1111 & 1100\\
	\hline
	\multicolumn{7}{|c|}{Tag} &
		
	Offset \\
	\multicolumn{7}{|c|}{1F296FF} &
	
	12 \\
	\hline
\end{tabular}
\end{frame}

\end{document}

%%%%%%%%%%%%%%%%%%%%%%%%%%%%%%%%%%%%%%%%%%%%%%%%%%%%%%%%%%%%%%%%%%%%%%%%%%%%%%%
