% Encoding: UTF-8
%%%%%%%%%%%%%%%%%%%%%%%%%%%%%%%%%%%%%%%%%%%%%%%%%%%%%%%%%%%%%%%%%%%%%%%%%%%%%%%

\documentclass[xcolor=pdftex,dvipsnames,table]{beamer}
\setbeamertemplate{navigation symbols}{}%remove navigation symbols
%%%%%%%%%%%%%%%%%%%%%%%%%%%%%%%%%%%%%%%%%%%%%%%%%%%%%%%%%%%%%%%%%%%%%%%%%%%%%%%

% Language settings
% IMPORTANT: If you change the language settings, make sure to run
% cleantex prior to any latex/pdflatex to remove all .aux files, else
% latex might fail!
\usepackage[english]{babel}
%\usepackage[german]{babel}
\usecolortheme{rose}
\usecolortheme{beaver}
\useinnertheme[shadow]{rounded}
\usecolortheme{dolphin}
%\usepackage{graphicx}
\usepackage[utf8]{inputenc}
%\usepackage{mathptmx}

\usepackage{setspace}
\usepackage{color}
\usepackage{listings}
	
	
%%%%%%%%%%%%%%%%%%%%%%%%%%%%%%%%%%%%%%%%%%%%%%%%%%%%%%%%%%%%%%%%%%%%%%%%%%%%%%%

% IMPORTANT/TODO: currently only PDF graphic files should be used, PNG
% or JPEG may harm the Adobe Reader colour palette and result in
% strange display colours.
%\DeclareGraphicsExtensions{.pdf}


%%%%%%%%%%%%%%%%%%%%%%%%%%%%%%%%%%%%%%%%%%%%%%%%%%%%%%%%%%%%%%%%%%%%%%%%%%%%%%%
%% Presentation main settings
%%%%%%%%%%%%%%%%%%%%%%%%%%%%%%%%%%%%%%%%%%%%%%%%%%%%%%%%%%%%%%%%%%%%%%%%%%%%%%%

\title{Systemnahe Informatik}
\subtitle{Übungsgruppe Xeon Phi}
\author{Dominik Walter}
\date{Sommersemester 2018}


%%%%%%%%%%%%%%%%%%%%%%%%%%%%%%%%%%%%%%%%%%%%%%%%%%%%%%%%%%%%%%%%%%

\begin{document}

\section*{Titlepage}
\begin{frame}
  \frametitle{\ }
  \titlepage
\end{frame}

%%%%%%%%%%%%%%%%%%%%%%%%%%%%%%%%%%%%%%%%%%%%%%%%%%%%%%%%%%%%%%%%%%%%%%%%%%%%%%%

\begin{frame}[fragile]
	\frametitle{Mutex}
	\begin{block}{Beispiel}
		\begin{lstlisting}	
// eventuell parallel
pthread_mutex_lock(&mutex);
// garantiert sequentiel
pthread_mutex_unlock(&mutex);
// eventuell parallel
		\end{lstlisting}
	\end{block}

	\begin{block}{Semantik}
		Warte bis kritischer Bereich betreten werden kann.
	\end{block}	

	\begin{block}{Verwendung}
		Schutz einer gemeinsamen Variablen / Datenstruktur
	\end{block}	
\end{frame}

\begin{frame}[fragile]
	\frametitle{Condition-Variable: Teil 1}
	\begin{block}{Beispiel}
		\begin{lstlisting}	
// eventuell parallel
pthread_mutex_lock(&mutex);
while (condition) {
	pthread_cond_wait(&cond, &mutex);
}
// garantiert sequentiel und !condition
pthread_mutex_unlock(&mutex);
// eventuell parallel
		\end{lstlisting}
	\end{block}
	
	\begin{block}{Semantik}
		Warte bis spezifische Bedingung erfüllt wurde.
	\end{block}	
	
	\begin{block}{Verwendung}
		Vermeidung von spinlocks (z.B. für queue::pop())
	\end{block}	
\end{frame}

\begin{frame}[fragile]
	\frametitle{Condition-Variable: Teil 2}
	\begin{block}{Beispiel}
		\begin{lstlisting}	
// eventuell parallel
pthread_mutex_lock(&mutex);
// garantiert sequentiel
condition = false;
pthread_mutex_unlock(&mutex);
// eventuell parallel
pthread_cond_signal(&cond);
		\end{lstlisting}
	\end{block}
	
	\begin{block}{Semantik}
		Bedingung wurde geändert und muss von allen Threads erneut überprüft werden.
	\end{block}	
\end{frame}

\begin{frame}[fragile]
	\frametitle{Semaphore: Teil 1}
	\begin{block}{Beispiel}
		\begin{lstlisting}	
// eventuell parallel
sem_wait(&sem);
// garantiert maximal n threads 
// (-> sem_init(&sem, 0, n))
sem_post(&sem);
// eventuell parallel
		\end{lstlisting}
	\end{block}
	
	\begin{block}{Semantik}
		Warte bis "kritischer" Bereich (n threads gleichzeitig) betreten werden kann.
	\end{block}	
	
	\begin{block}{Verwendung}
		Beschränkung der Parallelität
	\end{block}	
\end{frame}

\begin{frame}[fragile]
	\frametitle{Semaphore: Teil 2}
	
	\begin{columns}
		\begin{column}{0.52\textwidth}
			\begin{block}{Beispiel: Consumer-Thread}
			\begin{lstlisting}	
sem_wait(&sem);
//Ressource ist vorhanden
pthread_mutex_lock(&m);
y = queue.pop();
pthread_mutex_unlock(&m);
			\end{lstlisting}
			\end{block}
		\end{column}
		\begin{column}{0.52\textwidth}
		\begin{block}{Beispiel: Producer-Thread}
			\begin{lstlisting}	
pthread_mutex_lock(&m);
queue.push(x);
pthread_mutex_unlock(&m);
sem_post(&sem);
//Ressource ist vorhanden
			\end{lstlisting}
		\end{block}
		\end{column}
	\end{columns}
	
	\begin{block}{Semantik}
		Warte bis eine abstrakte Ressource verfügbar ist. Der eigentliche Zugriff muss allerdings separat geschützt werden!
	\end{block}	
	
	\begin{block}{Verwendung}
		Producer-Consumer Modell
	\end{block}	
\end{frame}

\begin{frame}[fragile]
	\frametitle{Barriere}
	\begin{block}{Beispiel}
		\begin{lstlisting}
// t_id: thread-ID	
// Phase 1: Globales Lesen
local = buffer[t_id + 1];
pthread_barrier_wait(&barrier);
// Phase 2: Lokales Schreiben
buffer[t_id] += local;
pthread_barrier_wait(&barrier);
		\end{lstlisting}
	\end{block}
	
	\begin{block}{Semantik}
		Warte bis n threads die Barriere erreicht haben.
	\end{block}	
	
	\begin{block}{Verwendung}
		Unterteilung in verschiedene Phasen (z.B. Bulk-Synchronisation)
	\end{block}	
\end{frame}

\end{document}

%%%%%%%%%%%%%%%%%%%%%%%%%%%%%%%%%%%%%%%%%%%%%%%%%%%%%%%%%%%%%%%%%%%%%%%%%%%%%%%
